\documentclass[a4paper,12pt]{article}
\usepackage[a4paper, hmargin={1.8cm, 1.8cm}, vmargin={1cm, 1cm}]{geometry}

\usepackage[style=alphabetic,sorting=nyt]{biblatex}
\addbibresource{sources.bib}

\usepackage{amsfonts}
\usepackage{amsmath}
\usepackage{amssymb}

\usepackage{graphicx}
\usepackage{xcolor}

\usepackage{tocloft}
\setcounter{tocdepth}{3}

\usepackage{hyperref}
\hypersetup{
  colorlinks=false,
  linkcolor=blue,
  filecolor=magenta,
  urlcolor=blue,
}
\usepackage[english]{babel}


\usepackage{lastpage}
\usepackage{fancyhdr}
\pagestyle{fancy}
\fancyhf{}
\renewcommand{\headrulewidth}{0pt}
\rfoot{Page \thepage\ of \pageref{LastPage}}

\setlength{\parskip}{1em}
\setlength{\parindent}{0px}

\title{Motivation of Bachelor Project PRE-DRAFT: A unifying framework for quanitifying the data propogation bottlenecks of graph representation learning methods - motivation}

\author{
  \color{red}  Mustafa Hekmat Al-Abdelamir\\
  \color{red}  Joshua Victor Niemelä\\
}
\date{\today}


\begin{document}


\maketitle



\section{Introduction}

Topological Deep Learning (TDL) is gaining traction as a novel approach~\cite{papamarkou_position:_2024} for Graph Representation Learning (GRL).
Leveraging topology, TDL has shown promising results in alleviating various limitations of Graph Neural Networks (GNNs)~\cite{horn_topological_2022}.

Two often-cited related\cite{giraldo_trade-off_2023} shortcomings in GNNs are over-smoothing and over-squashing.
Over-smoothing occurs when individual node features become washed out and too similar after multiple graph convolutions~\cite{li_deeper_2018}.
In message-passing, nodes send fixed-size messages to their neighbours, said neighbours aggregate the messages, update their features, then send out new messages and so on.
This process inevitably leads to information loss, as an increasingly large amount of information is compressed into fixed-size vectors.
This is known as over-smoothing~\cite{alon_bottleneck_2021}.

Still, perhaps because of the young nature of the field, there is limited theoretical foundation for quantifying these possible advantages.
This poses a problem in making quantitative comparisons between various architectures for GRL.
In this paper, we attempt to lay the foundations for various metrics to provide insights on over-squashing and over-smoothing and how they relate to the model's ability to learn.
To verify and support our theoretical foundations, we also run benchmarks against other approaches used for learning on graph data~\cite{horn_topological_2022}

\section{Data}
We will be using the QM9 dataset~\cite{blum}\cite{rupp} as real-world data to benchmark our models against
other models from the cited papers and to test the metrics we will develop.

We will also be creating and using synthetic datasets as this will give us more control over the data and allow us to create specific topological features and properties that we can then use to test the metrics and models.

\section{Methods}
\begin{itemize}
	\item  We will be using PyTorch Geometric to build our models and to replicate models cited in other papers as a comparison and empirical exploration of GRL.
	\item Docker will be used to containerise our experiments and replications to ensure reproducibility as well as make it easier to run experiments on different machines.
\end{itemize}

\section{Learning Objectives}


\begin{enumerate}
	\item Gain a solid understanding of the theoretical foundations of GNNs.
	\item Investigate the over-squashing problem in GNNs and how it affects the ability of the model to learn long-range dependencies~\cite{alon_bottleneck_2021}.
	\item Gain insights and understanding about TDL, how topology is leveraged for learning, and how it relates to the aforementioned bottlenecks~\cite{horn_topological_2022}.
	\item Construct generalisable metrics to quantify various geometric and topological properties of GNNs and the datasets they are trained on.
	\item Construct a model, for instance, a transformer or CCNN~\cite{tdlbook}, that can learn topological features in data and benchmark against non-topological approaches.
\end{enumerate}

\printbibliography

\end{document}

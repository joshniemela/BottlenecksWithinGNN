\documentclass[a4paper,12pt]{article}


\usepackage[a4paper, hmargin={1.8cm, 1.8cm}, vmargin={1cm, 1cm}]{geometry}

\usepackage[style=alphabetic,sorting=nyt]{biblatex}
\addbibresource{sources.bib}




\usepackage{amsfonts}
\usepackage{amsmath}
\usepackage{amssymb}

\usepackage{graphicx}  % For including images
\usepackage{xcolor}  % For a colorfull presentation

\usepackage{hyperref}

\usepackage{tocloft}
\setcounter{tocdepth}{3}


\usepackage{hyperref}
\hypersetup{
    colorlinks=false,
    linkcolor=blue,
    filecolor=magenta,
    urlcolor=blue,
}
\usepackage[english]{babel}


\usepackage{lastpage} % Gør s˚a du kan referere til fuldt sideantal i header/footer
\usepackage{fancyhdr} % headers and footers
\pagestyle{fancy} % layout-relateret
\fancyhf{} % skaber plads til header/footer
\renewcommand{\headrulewidth}{0pt}
\rfoot{Page \thepage\ of \pageref{LastPage}}

\setlength{\parskip}{1em}
\setlength{\parindent}{0px}

\begin{document}
\title{Motivation of Bachelor Project PRE-DRAFT: A unifying framework for quanitifying the data propogation bottlenecks of graph representation learning methods - motivation}

\author{
\color{red}  Mustafa Hekmat Al-Abdelamir\\
\color{red}  Joshua Victor Niemelä\\
}

\date{\today}
\maketitle

\begin{document}
\maketitle

\section{Introduction}
To alleviate various limitations of Graph Neural Networks (GNNs) for use in Graph Representation Learning (GRL), Topological Deep Learning (TDL) is emerging as a promising field of machine learning. TDL has shown promising results in tackling two common issues in GNNs: over-smoothing, where node features become too similar after multiple graph convolutions, and over-squashing, where information from distant nodes is excessively compressed into fixed-size vectors, leading to information loss \cite{alon_bottleneck_2021}. Still, perhaps because of the young nature of the field, there is limited theoretical foundation for quantifying these possible advantages. This poses a problem in making quantitative comparisons between various architectures for GRL. In this paper, we attempt to lay the foundations for various metrics to provide insights on over-squashing and over-smoothing and how they relate to the model's ability to learn. To verify and support our theoretical foundations, we also run benchmarks against other approaches used for learning on graph data.

\section{Data}
We will be using the QM9, Benchmark dataset for graph classification and the Human Body segmentation datasets to benchmark our own implementations against other models cited in various papers.

Additionally, we will be constructing synthetic datasets to compare the expressivity between various models for the task of identifying various topological structures in graphs.

\section{Methods}
We will be using Pytorch Geometric to build our models and to replicate models cited in other papers as a comparison and empirical exploration of GRL.

\section{Learning Objectives}


\begin{enumerate}
    \item Gain a solid understanding of the theoretical foundations of GNNs.
    \item Investigate the oversquashing problem in GNNs and how it affects the ability of the model to learn long-range dependencies\cite{alon_bottleneck_2021}.
    \item Gain insights and understanding about TDL, how topology is leveraged for learning, and how it relates to the aforementioned bottlenecks\cite{horn_topological_2022}.
    \item Construct generalisable metrics to quantify various geometric and topological properties of GNNs and the datasets they are trained on.
    \item Construct a model, for instance, a transformer or CCNN\cite{tdlbook}, that can learn topological features in data and benchmark against non-topological approaches.
\end{enumerate}

\printbibliography

\end{document}



\thispagestyle{empty}

\newpage % Start a new page after the table of contents
\setcounter{page}{1}


\printbibliography[heading=bibintoc]

\end{document}
